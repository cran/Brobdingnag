% -*- mode: noweb; noweb-default-code-mode: R-mode; -*-

\documentclass[a4paper]{article}

\author{Robin K. S. Hankin}
\title{A step-by-step guide to writing a simple package that uses S4 methods: a ``hello world'' example}
%\VignetteIndexEntry{Brobdingnag: a ``hello world'' package using S4 methods}

\usepackage{a4wide}

\usepackage{/usr/local/lib/R/share/texmf/Sweave}
\begin{document}
\maketitle

\section{Introduction}

This vignette proves that it is possible for a `normal'
person\footnote{That is, someone without super powers (such as might
manifest themselves after being bitten by a radioactive member of
R-core, for example)} to write a package using S4 methods.  It gives a
step-by-step guide to creating a package that contains two S4 classes
(brobs and glubs) and a bunch of basic utilities for manipulating
them.

This document focuses on the S4 aspects of the package.  For an
overview of the mathematical properties of Brobdingnagian numbers,
their potential and their limitations, see the {\tt .Rd} files and the
entry in Rnews (forthcoming).

If you like this vignette and package, and find it useful, let me
know.  If there is anything wrong with it, let me know.

I would not recommend that anyone uses S4 unless there is a good
reason for it (many of my packages use S3 methods which I found to be
perfectly adequate for my needs).  Reasons for using S4 might include
a package having a large number of object classes that have a
complicated hierarchical structure, or a complicated set of methods
that interact with the object classes in a complicated manner.

In the package, brobs are dealt with in {\tt brob.R}, and glubs are
treated in {\tt glub.R} which appropriately generalizes all the {\tt
brob} functionality.

This document could not have been prepared (and should not be read)
without consulting the following resources:
\begin{itemize}
\item John M. Chambers (1998), {\it Programming with Data}.  New York: Springer,
     ISBN 0-387-98503-4 (The Green Book).
\item W. N. Venables and B. D. Ripley (2000), {\it S Programming}.  Springer,
     ISBN 0-387-98966-8.
\item John Chambers (2006).  {\tt How S4 methods work} (available on CRAN).
\end{itemize}

\subsection{Overview}


The idea of {\tt Brobdingnag} package is simple: the IEEE
representation for floating point numbers cannot represent numbers
larger than about~$1.8\times 10^{308}$.  The package represents a
number by the natural logarithm of its magnitude, and also stores a
Boolean value indicating its sign.  Objects so stored have class {\tt
brob}; complex numbers may be similarly represented and have class
{\tt glub}.

With this scheme, multiplication is easy but addition is hard.  The
basic identity is:

\[
\log(e^x+e^y) = 
\left\{
\begin{array}{cc}
x+\log\left(1+e^{y-x}\right)\qquad &\mbox{if~$x>y$}\\
y+\log\left(1+e^{x-y}\right)\qquad &\mbox{otherwise}
\end{array}
\right.
\]

In practice this gets more complicated as one has to keep track of the
sign; and special dispensation is needed for zero ($=e^{-\infty}$).

One can thus deal with numbers up to about~$e^{1.9\times
10^{308}}\simeq 10^{7.8\times 10^{307}}$, although at this outer limit
accuracy is pretty poor.



\section{Class definition}

The first thing we need to do is to define the {\tt brob} class.  This
uses the {\tt setClass()} function:

\begin{Schunk}
\begin{Sinput}
> setClass("swift", representation = "VIRTUAL")
> setClass("brob", representation = representation(x = "numeric", 
+     positive = "logical"), prototype = list(x = numeric(), positive = logical()), 
+     contains = "swift")
\end{Sinput}
\end{Schunk}

It is simpler to ignore the first call to {\tt setClass()} here; for
reasons that will become apparent when discussing the {\tt c()}
function, one needs a virtual class that contains class brob and glub.
To understand virtual classes, see section~\ref{logicSection}.

The second call to {\tt setClass()} is more germane.  Let's take this
apart, argument by argument.  The first argument, {\tt
representation}, specifies ``the slots that the new class should have
and/or other classes that this class extends.  Usually a call to the
`representation' function''.  The helppage for {\tt representation}
gives a few further details.  Thus this argument specifies two
`slots': one for the value and one for the sign.  These are specified
to be numeric and logical (NB: not Boolean) respectively.

The second argument, {\tt prototype}, specifies default data for the
slots.  This kicks in when defining a zero-length brob; an example
would be extracting {\tt x[FALSE]} where {\tt x} is a brob.

The third argument, {\tt contains}, tells R that class {\tt swift}
(which was specified to be virtual), has {\tt brob} as a subclass.
We will need this later when we start to deal with {\tt glub}s, which
are also a subclass of {\tt swift}.

Let's use it:

\begin{Schunk}
\begin{Sinput}
> new("brob", x = 1:10, positive = rep(TRUE, 10))
\end{Sinput}
\begin{Soutput}
An object of class "brob"
Slot "x":
 [1]  1  2  3  4  5  6  7  8  9 10

Slot "positive":
 [1] TRUE TRUE TRUE TRUE TRUE TRUE TRUE TRUE TRUE TRUE
\end{Soutput}
\end{Schunk}

Notes:
\begin{itemize}
\item Function {\tt new()} is the {\em only} way to create objects
  of class brob.  So, any object of class brob {\em must} have been
  created with function {\tt new()}.  This is part of what the
  ``formal'' tag for S4 means\footnote{Compare S3, in which I can say

  {\tt a <- 1:10; class(a) <- "lilliput"}}.
\item Function {\tt new()} requires its arguments to be named, and no
  partial argument matching is performed.
\item Function {\tt new()} is not intended for the user.   It's too
  picky and difficult.  To create new brobs, we need some more
  friendly functions---{\tt as.brob()} and {\tt brob()}---discussed
  below.
\item There is, as yet, no print method, so the form of the object
  printed to the screen is less than ideal.
\end{itemize}

\subsection{Validity methods}

Now, an optional step is to define a function that tests whether the
arguments passed to {\tt new()} are acceptable.  As it stands, the
following code:
\begin{Schunk}
\begin{Sinput}
> new("brob", x = 1:10, positive = c(TRUE, FALSE, FALSE))
\end{Sinput}
\begin{Soutput}
An object of class "brob"
Slot "x":
 [1]  1  2  3  4  5  6  7  8  9 10

Slot "positive":
[1]  TRUE FALSE FALSE
\end{Soutput}
\end{Schunk}

\noindent
will not return an error, but is not acceptable because the arguments
are different lengths (and will not recycle neatly).

So, we define a validity method:

\begin{Schunk}
\begin{Sinput}
> .Brob.valid <- function(object) {
+     len <- length(object@positive)
+     if (len != length(object@x)) {
+         return("length mismatch")
+     }
+     else {
+         return(TRUE)
+     }
+ }
\end{Sinput}
\end{Schunk}

Advice on the form of the validity-testing function---here {\tt
.Brob.valid()}---is given in the help page for {\tt setValidity}:
``The method should be a function of one object that returns `TRUE' or
a description of the non-validity''.  Examples are given in section
7.1.6 of the Green Book.

In this package, I define a whole bunch of functions whose name starts
with {\tt .Brob.}; these are internal and not intended for the user.
They are also not documented.

So now we have a function, {\tt .Brob.valid()}, that checks whether
its argument has slots of the same length.  We need to tell R that
this function should be invoked every time a {\tt brob} is created.
Function {\tt setValidity()} does this:

\begin{Schunk}
\begin{Sinput}
> setValidity("brob", .Brob.valid)
\end{Sinput}
\begin{Soutput}
Slots:
                        
Name:         x positive
Class:  numeric  logical

Extends: "swift"
\end{Soutput}
\end{Schunk}

Thus, from now on [ie after the above call to {\tt setValidity()}],
when calling {\tt new("brob", ...)} the two arguments {\tt x} and {\tt
positive} must be the same length: recycling is not carried out.

Functions like {\tt .Brob.valid()} that are user-unfriendly all have
names beginning with {\tt .Brob}.  These functions are there to help
the organization of the package and are not intended to be used by the
end-user.

Clever, user-friendly operations such as recycling are carried out in
the more user-friendly functions such as {\tt as.brob()}.

If one were to call {\tt new()} with arguments of differing lengths,
as in
\\
\\
{\tt 
new("brob",x=1:10,positive=TRUE)
}
\\
\\
then {\tt new()} would report the error message in function {\tt
  .Brob.valid()}, because the {\tt positive} argument had length~1 and
the {\tt x} was length~10; and the validity method {\tt .Brob.valid()}
requires both arguments to be the same length\footnote{Placing the
  above call in {\tt try()} and showing the error explicitly would
  cause the package to fail {\tt R CMD check}.}.

So now {\tt new()} works, but isn't exactly user-friendly: often one
would want the above call to recycle the second argument to length 10
to match the first.  This deficiency is remedied in the next section.

\section{Basic user-friendly functions to create brobs}

The basic, semi user-friendly function for creating brobs is {\tt
brob()}:

\begin{Schunk}
\begin{Sinput}
> "brob" <- function(x = double(), positive) {
+     if (missing(positive)) {
+         positive <- rep(TRUE, length(x))
+     }
+     if (length(positive) == 1) {
+         positive <- rep(positive, length(x))
+     }
+     new("brob", x = as.numeric(x), positive = positive)
+ }
\end{Sinput}
\end{Schunk}

Thus {\tt brob(x)} will return a number formally equal to~$e^x$.
Function {\tt brob()} does helpful things like assuming the user
desires positive numbers; it also carries out recycling:

\begin{Schunk}
\begin{Sinput}
> brob(1:10, FALSE)
\end{Sinput}
\begin{Soutput}
An object of class "brob"
Slot "x":
 [1]  1  2  3  4  5  6  7  8  9 10

Slot "positive":
 [1] FALSE FALSE FALSE FALSE FALSE FALSE FALSE FALSE FALSE FALSE
\end{Soutput}
\end{Schunk}

Note that {\tt brob()} isn't exactly terribly user-friendly: it's
confusing. {\tt brob(5)} returns a number formally equal to~$e^5$,
not~$5$.   This is documented in the help page, where the user is
encouraged to use function {\tt as.brob()} instead.

\section{Testing for brobs: an {\tt is.brob()} function}

Function {\tt is()} will test for an object being a brob:

\begin{Schunk}
\begin{Sinput}
> is(brob(1:5), "brob")
\end{Sinput}
\begin{Soutput}
[1] TRUE
\end{Soutput}
\end{Schunk}

(see {\tt help(is)} for more details) but a small package like this,
with only brobs and glubs to consider, could benefit from an S3-style
function {\tt is.brob()}.  This is easy to define:

\begin{Schunk}
\begin{Sinput}
> is.brob <- function(x) {
+     is(x, "brob")
+ }
> is.glub <- function(x) {
+     is(x, "glub")
+ }
\end{Sinput}
\end{Schunk}

now the user can just type {\tt is.brob(x)} to find out whether an
object is a brob\footnote{This approach is fine for a tiny package
like Brobdingnag, with only two or three classes.  However, in the
context of a more complicated package such as {\tt Matrix}, which uses
dozens of different classes in a complicated hierarchical structure,
one might prefer to type {\tt is(x,"dpoMatrix-class")} rather than
define a plethora of functions along the lines of {\tt
is.dpoMatrix-class()}.}.

We also define an {\tt is.glub()} function similarly.  So now we can
check for objects being brobs and glubs.

\section{Coercion: an {\tt as.brob()} function}

Next, some ways to coerce objects to class brob:

\begin{Schunk}
\begin{Sinput}
> "as.brob" <- function(x) {
+     if (is.brob(x)) {
+         return(x)
+     }
+     else if (is.complex(x)) {
+         warning("imaginary parts discarded")
+         return(Recall(Re(x)))
+     }
+     else if (is.glub(x)) {
+         warning("imaginary parts discarded")
+         return(Re(x))
+     }
+     else {
+         return(brob(log(abs(x)), x >= 0))
+     }
+ }
\end{Sinput}
\end{Schunk}

So now we can coerce objects of various classes to brobs\footnote{The
recommended way, appropriate for a complicated package such as Matrix,
would be to execute {\tt setAs("numeric", "brob",
.Brob.numeric\_to\_brob)} and {\tt setAs("complex", "brob",
.Brob.complex\_to\_brob)}.  Then, if {\tt x} is numeric, {\tt
as(x,"brob")} would return the appropriate brob via {\tt
.Brob.numeric\_to\_brob()}; we could then make {\tt as.brob()} a
generic with {\tt setGeneric()} and define methods for it.  Doing it
this way would save function {\tt as.brob()} having to test for its
argument being a brob [carried out in the first few lines of {\tt
as.brob()}] because {\tt as()} has such a test built in, implicitly.}.
The {\em only} way to create a brob is to use {\tt new()}, and the
{\em only} function that calls this is {\tt brob()}.  And {\tt
as.brob()} calls this.

Note the user-friendliness of {\tt as.brob()}.  It takes numerics,
brobs, and glubs (which give a warning).

Check  {\tt as.brob()}:

\begin{Schunk}
\begin{Sinput}
> as.brob(1:10)
\end{Sinput}
\begin{Soutput}
An object of class "brob"
Slot "x":
 [1] 0.0000000 0.6931472 1.0986123 1.3862944 1.6094379 1.7917595 1.9459101
 [8] 2.0794415 2.1972246 2.3025851

Slot "positive":
 [1] TRUE TRUE TRUE TRUE TRUE TRUE TRUE TRUE TRUE TRUE
\end{Soutput}
\end{Schunk}

\section{Coercion: an {\tt as.numeric()} function}

Now we need some methods to coerce brobs to numeric.  This is a
two-stage process.


\begin{Schunk}
\begin{Sinput}
> setAs("brob", "numeric", function(from) {
+     out <- exp(from@x)
+     out[!from@positive] <- -out[!from@positive]
+     return(out)
+ })
\end{Sinput}
\end{Schunk}

This call to {\tt setAs()} makes {\tt as(x,"numeric")} carry out the
function passed as the third argument when given a brob. 

But in this package, the user isn't supposed to type {\tt
as(x,"numeric")}: the user is supposed to type {\tt
as.numeric(x)}\footnote{Users can be expected to be familiar with
functions such as {\tt as.numeric()} and {\tt as.complex()}, which is
why the Brobdingnag package recommends this form.  However, this might
not be appropriate for a more complicated package such as Matrix
because, to quote Martin Maechler, ``it seems very ugly to define more
than very few {\tt as.FOO()} methods, and very natural to work with
{\tt as(*, "FOO")} [constructions]''.  Of course, there's nothing to
stop a user typing {\tt as(x,"numeric")} if they wish.}.  To
accomplish this, we have to tell R that function {\tt as.numeric()}
should execute {\tt as(x,"numeric")} when given a {\tt brob}.  This is
done by calling function {\tt setMethod()}:


\begin{Schunk}
\begin{Sinput}
> setMethod("as.numeric", signature(x = "brob"), function(x) {
+     as(x, "numeric")
+ })
\end{Sinput}
\end{Schunk}

We similarly need to make {\tt as.complex()} work for brobs:
\begin{Schunk}
\begin{Sinput}
> setAs("brob", "complex", function(from) {
+     return(as.numeric(from) + (0+0i))
+ })
> setMethod("as.complex", signature(x = "brob"), function(x) {
+     as(x, "complex")
+ })
\end{Sinput}
\end{Schunk}

We'll need similar methods for glubs too.


Better check:



\begin{Schunk}
\begin{Sinput}
> x <- as.brob(1:4)
> x
\end{Sinput}
\begin{Soutput}
An object of class "brob"
Slot "x":
[1] 0.0000000 0.6931472 1.0986123 1.3862944

Slot "positive":
[1] TRUE TRUE TRUE TRUE
\end{Soutput}
\begin{Sinput}
> as.numeric(x)
\end{Sinput}
\begin{Soutput}
[1] 1 2 3 4
\end{Soutput}
\end{Schunk}

So that works.

\section{Print methods}
Print methods are not strictly necessary, but make using the package
much easier.  First, a helper function:


\begin{Schunk}
\begin{Sinput}
> .Brob.print <- function(x, digits = 5) {
+     noquote(paste(c("-", "+")[1 + x@positive], "exp(", signif(x@x, 
+         digits), ")", sep = ""))
+ }
\end{Sinput}
\end{Schunk}

Then an S3 method:

\begin{Schunk}
\begin{Sinput}
> print.brob <- function(x, ...) {
+     jj <- .Brob.print(x, ...)
+     print(jj)
+     return(invisible(jj))
+ }
\end{Sinput}
\end{Schunk}

And finally a call to {\tt setMethod()}:

\begin{Schunk}
\begin{Sinput}
> setMethod("show", "brob", function(object) {
+     print.brob(object)
+ })
\end{Sinput}
\end{Schunk}

This two-stage methodology is recommended in the Venables and Ripley.
The {\tt .Brob.print()} function does the hard work.  Example of it in
use:

\begin{Schunk}
\begin{Sinput}
> as.brob(1:4)
\end{Sinput}
\begin{Soutput}
[1] +exp(0)       +exp(0.69315) +exp(1.0986)  +exp(1.3863) 
\end{Soutput}
\end{Schunk}

See how the brob object is printed out nicely, and with no special
effort required of the user.


\section{Get and Set methods}

To be anal retentive about things, one should define {\tt C++} style
accessor functions as follows:

\begin{Schunk}
\begin{Sinput}
> setGeneric("getX", function(x) {
+     standardGeneric("getX")
+ })
> setGeneric("getP", function(x) {
+     standardGeneric("getP")
+ })
> setMethod("getX", "brob", function(x) {
+     x@x
+ })
> setMethod("getP", "brob", function(x) {
+     x@positive
+ })
\end{Sinput}
\end{Schunk}

but in practice I just use {\tt @} to access the slots.  These are
just here for good form's sake.


\section{Length}

Now a length:

\begin{Schunk}
\begin{Sinput}
> setMethod("length", "brob", function(x) {
+     length(x@x)
+ })
\end{Sinput}
\end{Schunk}

\section{Extracting elements of a vector}

Next thing is to define some methods for extraction.  This is done
with {\tt setMethod()} for extraction, and {\tt setReplaceMethod()}
for replacement.

\begin{Schunk}
\begin{Sinput}
> setMethod("[", "brob", function(x, i, j, drop) {
+     if (!missing(j)) {
+         warning("second argument to extractor function ignored")
+     }
+     brob(x@x[i], x@positive[i])
+ })
\end{Sinput}
\end{Schunk}

See how the third argument to {\tt setMethod()} is a function whose
arguments are the same as those to {\tt  "["() %]
}.  Argument {\tt j} {\em must} be there otherwise one gets a
  signature error.  I've put in a warning if a second argument that
  might be interpreted as {\tt j} is given.

Now a method for replacement.  This is a call to {\tt setReplaceMethod()}:


\begin{Schunk}
\begin{Sinput}
> setReplaceMethod("[", signature(x = "brob"), function(x, i, j, 
+     value) {
+     if (!missing(j)) {
+         warning("second argument to extractor function ignored")
+     }
+     jj.x <- x@x
+     jj.pos <- x@positive
+     if (is.brob(value)) {
+         jj.x[i] <- value@x
+         jj.pos[i] <- value@positive
+         return(brob(x = jj.x, positive = jj.pos))
+     }
+     else {
+         x[i] <- as.brob(value)
+         return(x)
+     }
+ })
\end{Sinput}
\end{Schunk}


See how the replacement function tests for the replacement value being
a brob and acts accordingly.

\section{Concatenation function {\tt cbrob()}}
\label{cbrob}

It is not possible to make {\tt c()} behave as expected for
brobs\footnote{The ideas in this section are entirely due to John
Chambers, who kindly replied to a question of mine on the R-devel
email list} (that is, if any of its arguments are brobs, to coerce all
its arguments to brobs and then concatenate).

However, it is possible to define a function {\tt cbrob()} that does
the job.  This has to be done in several stages.

First we define another user-unfriendly helper function {\tt
.Brob.cPair()} which takes two arguments, coerces them to brobs, and
concatenates them:

\begin{Schunk}
\begin{Sinput}
> .Brob.cPair <- function(x, y) {
+     x <- as.brob(x)
+     y <- as.brob(y)
+     brob(c(x@x, y@x), c(x@positive, y@positive))
+ }
\end{Sinput}
\end{Schunk}

This is just {\tt c()} for the two slots separately.  The idea is that
function {\tt .Brob.cPair()} takes two arguments; both are coerced to
brobs and it returns the concatenated vector of brobs.

Now, we need to set up a (user-unfriendly) generic function {\tt
.cPair()}:

\begin{Schunk}
\begin{Sinput}
> setGeneric(".cPair", function(x, y) {
+     standardGeneric(".cPair")
+ })
\end{Sinput}
\end{Schunk}

Function {\tt .cPair()} is not substantive (sic): it exists purely in
order to be a generic function that dispatches to {\tt .Brob.cPair()}.

Now we use {\tt setMethod()} to organize the dispatch:


\begin{Schunk}
\begin{Sinput}
> setMethod(".cPair", c("brob", "brob"), function(x, y) {
+     .Brob.cPair(x, y)
+ })
> setMethod(".cPair", c("brob", "ANY"), function(x, y) {
+     .Brob.cPair(x, as.brob(y))
+ })
> setMethod(".cPair", c("ANY", "brob"), function(x, y) {
+     .Brob.cPair(as.brob(x), y)
+ })
> setMethod(".cPair", c("ANY", "ANY"), function(x, y) {
+     c(x, y)
+ })
\end{Sinput}
\end{Schunk}


The four calls are necessary for the four different signatures that
might be encountered.  Note the {\tt ANY} class in the second, third,
and fourth call.  Thus if someone wants to write a new class of object
(a lugg, say), and wants to concatenate luggs with a brob, this will
work provided that they use {\tt setAs()} to make {\tt as.brob()}
coerce correctly for lugg objects.  The method used here allows this
to be done without any changes to the Brobdingnag package.

The final stage is the definition of {\tt cbrob()}, a user-friendly
wrapper for the above stuff:

\begin{Schunk}
\begin{Sinput}
> "cbrob" <- function(x, ...) {
+     if (nargs() < 3) 
+         .cPair(x, ...)
+     else .cPair(x, Recall(...))
+ }
\end{Sinput}
\end{Schunk}


Note the recursive definition.  If {\tt cbrob()} is called with {\em
any} set of arguments that include a brob anywhere, this will result
in the whole lot being coerced to brobs [by {\tt .Brob.cPair()}].
Which is what we want (although glubs will require more work).


Just test this:
\begin{Schunk}
\begin{Sinput}
> a <- 1:3
> b <- as.brob(1e+100)
> cbrob(a, a, b, a)
\end{Sinput}
\begin{Soutput}
 [1] +exp(0)       +exp(0.69315) +exp(1.0986)  +exp(0)       +exp(0.69315)
 [6] +exp(1.0986)  +exp(230.26)  +exp(0)       +exp(0.69315) +exp(1.0986) 
\end{Soutput}
\end{Schunk}

So it worked: everything was coerced to a brob because of the single
object of class brob in the call.


\section{Maths}

The math group generic functions are set with function {\tt
  setMethod()}.  But, before this can be called, function {\tt sqrt()}
needs a specific brob method:


\begin{Schunk}
\begin{Sinput}
> setMethod("sqrt", "brob", function(x) {
+     brob(ifelse(x@positive, x@x/2, NaN), TRUE)
+ })
\end{Sinput}
\end{Schunk}

Just check that:
\begin{Schunk}
\begin{Sinput}
> sqrt(brob(4))
\end{Sinput}
\begin{Soutput}
[1] +exp(2)
\end{Soutput}
\end{Schunk}

With these out of the way we can use {\tt setMethod()} to define the 
appropriate functions in the math group generic:

\begin{Schunk}
\begin{Sinput}
> setMethod("Math", "brob", function(x) {
+     switch(.Generic, abs = brob(x@x), log = {
+         out <- x@x
+         out[!x@positive] <- NaN
+         out
+     }, exp = brob(x), cosh = {
+         (brob(x) + brob(-x))/2
+     }, sinh = {
+         (brob(x) - brob(-x))/2
+     }, acos = , acosh = , asin = , asinh = , atan = , atanh = , 
+         cos = , sin = , tan = , tanh = , trunc = callGeneric(as.numeric(x)), 
+         lgamma = , cumsum = , gamma = , ceiling = , floor = as.brob(callGeneric(as.numeric(x))), 
+         stop(paste(.Generic, "not allowed on Brobdingnagian numbers")))
+ })
\end{Sinput}
\end{Schunk}

See how the third argument to {\tt setMethod()} is a function.  This
function has access to {\tt .Generic}, in addition to {\tt x} and uses
it to decide which operation to perform.

See how functions {\tt acos()} to {\tt trunc()} just drop through to
{\tt callGeneric(as.numeric(x))}.  See also the method for {\tt
log()}, which uses facts about brobs not known to S4.

Just a quick check:

\begin{Schunk}
\begin{Sinput}
> sin(brob(4))
\end{Sinput}
\begin{Soutput}
[1] -0.928768
\end{Soutput}
\end{Schunk}

So that works.

\section{Operations}

Now we need to make sure that {\tt brob(1) + brob(3)} works: the
operations {\tt +, -, *, /} must work as expected.  This is hard.

First step: define some user-unfriendly functions that carry out the
operations.  For example, function {\tt .Brob.negative()} simply
returns the negative of a brob.  These functions are not for the user.


\begin{Schunk}
\begin{Sinput}
> .Brob.negative <- function(e1) {
+     brob(e1@x, !e1@positive)
+ }
> .Brob.ds <- function(e1, e2) {
+     xor(e1@positive, e2@positive)
+ }
> .Brob.add <- function(e1, e2) {
+     e1 <- as.brob(e1)
+     e2 <- as.brob(e2)
+     jj <- rbind(e1@x, e2@x)
+     x1 <- jj[1, ]
+     x2 <- jj[2, ]
+     out.x <- double(length(x1))
+     jj <- rbind(e1@positive, e2@positive)
+     p1 <- jj[1, ]
+     p2 <- jj[2, ]
+     out.pos <- p1
+     ds <- .Brob.ds(e1, e2)
+     ss <- !ds
+     out.x[ss] <- pmax(x1[ss], x2[ss]) + log1p(+exp(-abs(x1[ss] - 
+         x2[ss])))
+     out.x[ds] <- pmax(x1[ds], x2[ds]) + log1p(-exp(-abs(x1[ds] - 
+         x2[ds])))
+     out.x[(x1 == -Inf) & (x2 == -Inf)] <- -Inf
+     out.pos <- p1
+     out.pos[ds] <- xor((x1[ds] > x2[ds]), (!p1[ds]))
+     return(brob(out.x, out.pos))
+ }
> .Brob.mult <- function(e1, e2) {
+     e1 <- as.brob(e1)
+     e2 <- as.brob(e2)
+     return(brob(e1@x + e2@x, !.Brob.ds(e1, e2)))
+ }
> .Brob.power <- function(e1, e2) {
+     stopifnot(is.brob(e1) | is.brob(e2))
+     if (is.brob(e2)) {
+         return(brob(log(e1) * brob(e2@x), TRUE))
+     }
+     else {
+         s <- as.integer(2 * e1@positive - 1)
+         return(brob(e1@x * as.brob(e2), (s^as.numeric(e2)) > 
+             0))
+     }
+ }
> .Brob.inverse <- function(b) {
+     brob(-b@x, b@positive)
+ }
\end{Sinput}
\end{Schunk}

Note the complexity of {\tt .Brob.add()}.  This is hard because logs
are good at multiplying but bad at adding [{\tt ss} and {\tt ds} mean
``same sign'' and ``different sign'' respectively].

The first step is to make sure that the unary operators {\tt +} and
{\tt -} work.  We do this by a call to {\tt setMethod()}:


\begin{Schunk}
\begin{Sinput}
> setMethod("Arith", signature(e1 = "brob", e2 = "missing"), function(e1, 
+     e2) {
+     switch(.Generic, "+" = e1, "-" = .Brob.negative(e1), stop(paste("Unary operator", 
+         .Generic, "not allowed on Brobdingnagian numbers")))
+ })
\end{Sinput}
\end{Schunk}

Note the second argument is {\tt signature(e1 = "brob",
 e2="missing")}: this effectively restricts the scope to unary
 operators.  The {\tt switch()} statement only allows the + and the -.\


Check:

\begin{Schunk}
\begin{Sinput}
> -brob(5)
\end{Sinput}
\begin{Soutput}
[1] -exp(5)
\end{Soutput}
\end{Schunk}

So that works.

Next step, another user-unfriendly helper function that does the dirty work:


\begin{Schunk}
\begin{Sinput}
> .Brob.arith <- function(e1, e2) {
+     switch(.Generic, "+" = .Brob.add(e1, e2), "-" = .Brob.add(e1, 
+         .Brob.negative(as.brob(e2))), "*" = .Brob.mult(e1, e2), 
+         "/" = .Brob.mult(e1, .Brob.inverse(as.brob(e2))), "^" = .Brob.power(e1, 
+             e2), stop(paste("binary operator \"", .Generic, "\" not defined for Brobdingnagian numbers")))
+ }
\end{Sinput}
\end{Schunk}


And now we can call {\tt setMethod()}:

\begin{Schunk}
\begin{Sinput}
> setMethod("Arith", signature(e1 = "brob", e2 = "ANY"), .Brob.arith)
> setMethod("Arith", signature(e1 = "ANY", e2 = "brob"), .Brob.arith)
> setMethod("Arith", signature(e1 = "brob", e2 = "brob"), .Brob.arith)
\end{Sinput}
\end{Schunk}

Better check it:

\begin{Schunk}
\begin{Sinput}
> 1e+100 + as.brob(10)^100
\end{Sinput}
\begin{Soutput}
[1] +exp(230.95)
\end{Soutput}
\end{Schunk}



\section{Comparison}


This is pretty much the same as the others.  First, some
user-unfriendly helper functions:

\begin{Schunk}
\begin{Sinput}
> .Brob.equal <- function(e1, e2) {
+     (e1@x == e2@x) & (e1@positive == e2@positive)
+ }
> .Brob.greater <- function(e1, e2) {
+     jj.x <- rbind(e1@x, e2@x)
+     jj.p <- rbind(e1@positive, e2@positive)
+     ds <- .Brob.ds(e1, e2)
+     ss <- !ds
+     greater <- logical(length(ss))
+     greater[ds] <- jj.p[1, ds]
+     greater[ss] <- jj.p[1, ss] & (jj.x[1, ss] > jj.x[2, ss])
+     return(greater)
+ }
\end{Sinput}
\end{Schunk}

These are the fundamental ones.  We can now define another
all-encompassing user-unfriendly function:

\begin{Schunk}
\begin{Sinput}
> ".Brob.compare" <- function(e1, e2) {
+     e1 <- as.brob(e1)
+     e2 <- as.brob(e2)
+     switch(.Generic, "==" = .Brob.equal(e1, e2), "!=" = !.Brob.equal(e1, 
+         e2), ">" = .Brob.greater(e1, e2), "<" = !.Brob.greater(e1, 
+         e2) & !.Brob.equal(e1, e2), ">=" = .Brob.greater(e1, 
+         e2) | .Brob.equal(e1, e2), "<=" = !.Brob.greater(e1, 
+         e2) | .Brob.equal(e1, e2), stop(paste(.Generic, "not supported for Brobdingnagian numbers")))
+ }
\end{Sinput}
\end{Schunk}

See how this function coerces both arguments to brobs.
Now the call to {\tt setMethod()}:

\begin{Schunk}
\begin{Sinput}
> setMethod("Compare", signature(e1 = "brob", e2 = "ANY"), .Brob.compare)
> setMethod("Compare", signature(e1 = "ANY", e2 = "brob"), .Brob.compare)
> setMethod("Compare", signature(e1 = "brob", e2 = "brob"), .Brob.compare)
\end{Sinput}
\end{Schunk}

Better check:

\begin{Schunk}
\begin{Sinput}
> as.brob(10) < as.brob(11)
\end{Sinput}
\begin{Soutput}
[1] TRUE
\end{Soutput}
\begin{Sinput}
> as.brob(10) <= as.brob(10)
\end{Sinput}
\begin{Soutput}
[1] TRUE
\end{Soutput}
\end{Schunk}

So that works.

\section{Logic}
\label{logicSection}


(The material in this section works in R-2.4.1, but not R-2.4.0).

First a helper function:

\begin{Schunk}
\begin{Sinput}
> .Brob.logic <- function(e1, e2) {
+     stop("No logic currently implemented for Brobdingnagian numbers")
+ }
\end{Sinput}
\end{Schunk}


Now the calls to {\tt setMethod()}:


\begin{Schunk}
\begin{Sinput}
> setMethod("Logic", signature(e1 = "swift", e2 = "ANY"), .Brob.logic)
> setMethod("Logic", signature(e1 = "ANY", e2 = "swift"), .Brob.logic)
> setMethod("Logic", signature(e1 = "swift", e2 = "swift"), .Brob.logic)
\end{Sinput}
\end{Schunk}

Note that the signatures specify {\tt swift} objects, so that glubs
will be handled correctly too in one fell swoop. Note the third call
to {\tt setMethod()}: without this, a call to {\tt Logic()} with
signature {\tt c("swift","swift")} would be ambiguous as it might be
interpreted as {\tt c("swift","ANY")} or {\tt c("ANY","swift")}.

Here, class {\tt swift} extends {\tt brob} and {\tt glub}; so both
brobs and glubs {\em are} {\tt swift} objects.  But no object is a
``pure'' {\tt swift}; it's either a brob or a glub.  This is useful
here because I might dream up some new class of objects that are
``like'' brobs in some way (for example, a class of objects whose
elements are quaternions with Brobdingnagian components) and it would
be nice to specify behaviour that is generic to brobs and glubs and
the new class of objects too.




\section{Miscellaneous generics}

We now have to tell R that certain functions are to be considered
generic.  The functions are {\tt max(), min(), range(), prod()}, and
{\tt sum()}.  The help page for (eg) {\tt max()} specifies that the
arguments must be numeric, and brobs aren't numeric.

In versions of R prior to 2.6-0 (I think), {\tt log()} needs to be
made a generic with {\tt setGeneric()}.  But, in versions 2.6-0 and
above, all the group generics (including {\tt log}) are primitive,
which means that the generic function is implicit and cannot be
changed.  This applies to the other group generics too ({\tt max},
{\tt min}, {\tt prod}, {\tt range} {\tt sum}).  So to work with both
types of R, one needs to check whether or not {\tt log} (or {\tt max}
or whatever) is generic before calling {\tt setGeneric()}.

\begin{Schunk}
\begin{Sinput}
> if (!isGeneric("log")) {
+     setGeneric("log", group = "Math")
+ }
\end{Sinput}
\end{Schunk}


\begin{Schunk}
\begin{Sinput}
> if (!isGeneric("sum")) {
+     setGeneric("max", function(x, ..., na.rm = FALSE) {
+         standardGeneric("max")
+     }, useAsDefault = function(x, ..., na.rm = FALSE) {
+         base::max(x, ..., na.rm = na.rm)
+     }, group = "Summary")
+     setGeneric("min", function(x, ..., na.rm = FALSE) {
+         standardGeneric("min")
+     }, useAsDefault = function(x, ..., na.rm = FALSE) {
+         base::min(x, ..., na.rm = na.rm)
+     }, group = "Summary")
+     setGeneric("range", function(x, ..., na.rm = FALSE) {
+         standardGeneric("range")
+     }, useAsDefault = function(x, ..., na.rm = FALSE) {
+         base::range(x, ..., na.rm = na.rm)
+     }, group = "Summary")
+     setGeneric("prod", function(x, ..., na.rm = FALSE) {
+         standardGeneric("prod")
+     }, useAsDefault = function(x, ..., na.rm = FALSE) {
+         base::prod(x, ..., na.rm = na.rm)
+     }, group = "Summary")
+     setGeneric("sum", function(x, ..., na.rm = FALSE) {
+         standardGeneric("sum")
+     }, useAsDefault = function(x, ..., na.rm = FALSE) {
+         base::sum(x, ..., na.rm = na.rm)
+     }, group = "Summary")
+ }
\end{Sinput}
\end{Schunk}

Now we need some more user-unfriendly helper functions:

\begin{Schunk}
\begin{Sinput}
> .Brob.max <- function(x, ..., na.rm = FALSE) {
+     p <- x@positive
+     val <- x@x
+     if (any(p)) {
+         return(brob(max(val[p])))
+     }
+     else {
+         return(brob(min(val), FALSE))
+     }
+ }
> .Brob.prod <- function(x) {
+     p <- x@positive
+     val <- x@x
+     return(brob(sum(val), (sum(p)%%2) == 0))
+ }
> .Brob.sum <- function(x) {
+     .Brob.sum.allpositive(x[x > 0]) - .Brob.sum.allpositive(-x[x < 
+         0])
+ }
> .Brob.sum.allpositive <- function(x) {
+     if (length(x) < 1) {
+         return(as.brob(0))
+     }
+     val <- x@x
+     p <- x@positive
+     mv <- max(val)
+     return(brob(mv + log1p(sum(exp(val[-which.max(val)] - mv))), 
+         TRUE))
+ }
\end{Sinput}
\end{Schunk}

Note the final function that sums its arguments, which are all assumed
to be positive, in an intelligent, accurate, and efficient manner.  No
checking is done (this is not a user-friendly function!).

We can define {\tt .Brob.sum()} in terms of this: return the
difference between the sum of the positive arguments and and the sum
of minus the negative arguments.

\begin{Schunk}
\begin{Sinput}
> setMethod("Summary", "brob", function(x, ..., na.rm = FALSE) {
+     switch(.Generic, max = .Brob.max(x, ..., na.rm = na.rm), 
+         min = -.Brob.max(-x, ..., na.rm = na.rm), range = cbrob(min(x, 
+             na.rm = na.rm), max(x, na.rm = na.rm)), prod = .Brob.prod(x), 
+         sum = .Brob.sum(x), stop(paste(.Generic, "not allowed on Brobdingnagian numbers")))
+ })
\end{Sinput}
\end{Schunk}

Better check:

\begin{Schunk}
\begin{Sinput}
> sum(as.brob(1:100)) - 5050
\end{Sinput}
\begin{Soutput}
[1] -exp(-Inf)
\end{Soutput}
\end{Schunk}

showing acceptable accuracy.

\section{Examples of the package in use}

We can try to evaluate a factorial.
Stirling's approximation is~$n!\sim\sqrt{2\pi
  n}\,e^{-n}n^n$:

\begin{Schunk}
\begin{Sinput}
> stirling <- function(x) {
+     sqrt(2 * pi * x) * exp(-x) * x^x
+ }
\end{Sinput}
\end{Schunk}

And this should work seamlessly with Brobs:


\begin{Schunk}
\begin{Sinput}
> stirling(100)
\end{Sinput}
\begin{Soutput}
[1] 9.324848e+157
\end{Soutput}
\begin{Sinput}
> stirling(as.brob(100))
\end{Sinput}
\begin{Soutput}
[1] +exp(363.74)
\end{Soutput}
\end{Schunk}

And are they the same?

\begin{Schunk}
\begin{Sinput}
> as.numeric(stirling(100)/stirling(as.brob(100)))
\end{Sinput}
\begin{Soutput}
[1] 1
\end{Soutput}
\end{Schunk}

\ldots pretty near.  But the great advantage of Brobdingnagian numbers
is that they can handle numbers larger than the IEEE limit:

\begin{Schunk}
\begin{Sinput}
> stirling(1000)
\end{Sinput}
\begin{Soutput}
[1] NaN
\end{Soutput}
\begin{Sinput}
> stirling(as.brob(1000))
\end{Sinput}
\begin{Soutput}
[1] +exp(5912.1)
\end{Soutput}
\end{Schunk}

\noindent
and this is accurate to about 12 sig figs, which is accurate enough
for many purposes.  The number of sig figs decreases with
progressively larger numbers, essentially because increasing amounts
of floating point accuracy is gobbled up by storing the exponent of a
large number, and less is left for the mantissa.


\section*{Acknowledgments}

I gratefully acknowledge the help given to me by many members of the
R-help and R-devel lists, especially Martin Maechler, Brian Ripley,
and John Chambers.

\newpage

  \noindent
  Robin K. S. Hankin\\
  National Oceanography Centre, Southampton\\
  European Way\\
  Southampton SO14 3ZH\\
  United Kingdom\\
  E-mail: {\tt r.hankin@noc.soton.ac.uk}
  URL: {\tt http://www.noc.soton.ac.uk}


\end{document}
